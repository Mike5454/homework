\documentclass[11pt]{article}
\usepackage{mathtools,hyperref,booktabs,fullpage}
%\usepackage[amssymb,cdot]{SIunits}
%\usepackage[utopia]{mathdesign}     
\usepackage{xcolor}
\usepackage{amsmath}
\usepackage{amssymb}
\usepackage{hyperref}
\usepackage{longtable}
\usepackage{fullpage}
\usepackage{enumitem}
\setlist{nolistsep}

%\definecolor{lightgray}{gray}{0.93}

\pagestyle{empty}
\setlength\parindent{0pt}
\renewcommand{\thefootnote}{\fnsymbol{footnote}}
 
\makeatletter
\renewcommand\section{\@startsection{section}{1}{\z@}%
                                  {-3.5ex \@plus -1ex \@minus -.2ex}%
                                  {2.3ex \@plus.2ex}%
                                  {\normalfont\bfseries}}
\makeatother


\begin{document}

{\large
  \begin{center}
    {\bf ME 701 -- Development of Computer Applications In Mechanical Engineering \\ 
         Homework 8  -- Due 11/1/2016 \\
         \vspace{12pt}
         \textcolor{purple}{Last updated: \today}
    }
  \end{center}

\setlength{\unitlength}{1in}

}


\fbox{
  \parbox{0.95\textwidth}{
\textcolor{purple}{{\bf Instructions}}:   In class, we are using a pretty straightforward idea---a function evaluator---to
motivate several features of GUI's and PyQt.  Your job is to expand/modify the in-class 
exercises as specified below. 
  }
}

\vspace{12pt}

\fbox{
  \parbox{0.95\textwidth}{
\textcolor{purple}{{\bf Deliverables}}: 
  One TAR file  {\tt lastname\_firstname.tar} that contains a 
  Python file named {\tt lastname\_firstname.pyw} and a summary 
  file {\tt lastname\_firstname.pdf}.  All of the features request by
  Problems 1--3 should be implemented in {\tt lastname\_firstname.pyw}. )
  }
}



% Chapters 1,2,3



\section*{Problem 1}

During the first day, we developed a function/value/output 
GUI.  Replace the function box with a drop-down box of at least
three built-in functions, the first of which should be $\sin(x)$.
A fourth option should be an editable option so that the user may
still define a custom function.


\section*{Problem 2}

The original GUI allowed for a function $f(x)$ to be evaluated at a single
point $x$.  Extend the GUI to handle array-valued inputs.  Specifically, 
allow the users to enter 
\textcolor{purple}{{\tt 0, 1, 2, 3}} or 
\textcolor{purple}{{\tt np.linspace(0, 1, 4)}}
for $x$.  Moreover, you should provide a {\tt save as} feature that
saves the $x$ and $f(x)$ data to file.  



\section*{Problem 3}

During the third day, we saw how to plug into matplotlib in order to 
embed plots in our GUI.  Your job is to create a function plotter by
combining the results from the three lectures in class.  You should 
ensure that plots can be refreshed for new $f$ or $x$ values.  
Include a screenshot of the GUI for 
$\sin(x)$ over $x \in [0, \pi/2]$ using 100 points in your PDF.





\end{document}